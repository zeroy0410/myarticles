% Options for packages loaded elsewhere
\PassOptionsToPackage{unicode}{hyperref}
\PassOptionsToPackage{hyphens}{url}
%
\documentclass[
]{article}
\usepackage{lmodern}
\usepackage{amssymb,amsmath}
\usepackage{ifxetex,ifluatex}
\ifnum 0\ifxetex 1\fi\ifluatex 1\fi=0 % if pdftex
  \usepackage[T1]{fontenc}
  \usepackage[utf8]{inputenc}
  \usepackage{textcomp} % provide euro and other symbols
\else % if luatex or xetex
  \usepackage{unicode-math}
  \defaultfontfeatures{Scale=MatchLowercase}
  \defaultfontfeatures[\rmfamily]{Ligatures=TeX,Scale=1}
\fi
% Use upquote if available, for straight quotes in verbatim environments
\IfFileExists{upquote.sty}{\usepackage{upquote}}{}
\IfFileExists{microtype.sty}{% use microtype if available
  \usepackage[]{microtype}
  \UseMicrotypeSet[protrusion]{basicmath} % disable protrusion for tt fonts
}{}
\makeatletter
\@ifundefined{KOMAClassName}{% if non-KOMA class
  \IfFileExists{parskip.sty}{%
    \usepackage{parskip}
  }{% else
    \setlength{\parindent}{0pt}
    \setlength{\parskip}{6pt plus 2pt minus 1pt}}
}{% if KOMA class
  \KOMAoptions{parskip=half}}
\makeatother
\usepackage{xcolor}
\IfFileExists{xurl.sty}{\usepackage{xurl}}{} % add URL line breaks if available
\IfFileExists{bookmark.sty}{\usepackage{bookmark}}{\usepackage{hyperref}}
\hypersetup{
  hidelinks,
  pdfcreator={LaTeX via pandoc}}
\urlstyle{same} % disable monospaced font for URLs
\usepackage{color}
\usepackage{fancyvrb}
\newcommand{\VerbBar}{|}
\newcommand{\VERB}{\Verb[commandchars=\\\{\}]}
\DefineVerbatimEnvironment{Highlighting}{Verbatim}{commandchars=\\\{\}}
% Add ',fontsize=\small' for more characters per line
\newenvironment{Shaded}{}{}
\newcommand{\AlertTok}[1]{\textcolor[rgb]{1.00,0.00,0.00}{\textbf{#1}}}
\newcommand{\AnnotationTok}[1]{\textcolor[rgb]{0.38,0.63,0.69}{\textbf{\textit{#1}}}}
\newcommand{\AttributeTok}[1]{\textcolor[rgb]{0.49,0.56,0.16}{#1}}
\newcommand{\BaseNTok}[1]{\textcolor[rgb]{0.25,0.63,0.44}{#1}}
\newcommand{\BuiltInTok}[1]{#1}
\newcommand{\CharTok}[1]{\textcolor[rgb]{0.25,0.44,0.63}{#1}}
\newcommand{\CommentTok}[1]{\textcolor[rgb]{0.38,0.63,0.69}{\textit{#1}}}
\newcommand{\CommentVarTok}[1]{\textcolor[rgb]{0.38,0.63,0.69}{\textbf{\textit{#1}}}}
\newcommand{\ConstantTok}[1]{\textcolor[rgb]{0.53,0.00,0.00}{#1}}
\newcommand{\ControlFlowTok}[1]{\textcolor[rgb]{0.00,0.44,0.13}{\textbf{#1}}}
\newcommand{\DataTypeTok}[1]{\textcolor[rgb]{0.56,0.13,0.00}{#1}}
\newcommand{\DecValTok}[1]{\textcolor[rgb]{0.25,0.63,0.44}{#1}}
\newcommand{\DocumentationTok}[1]{\textcolor[rgb]{0.73,0.13,0.13}{\textit{#1}}}
\newcommand{\ErrorTok}[1]{\textcolor[rgb]{1.00,0.00,0.00}{\textbf{#1}}}
\newcommand{\ExtensionTok}[1]{#1}
\newcommand{\FloatTok}[1]{\textcolor[rgb]{0.25,0.63,0.44}{#1}}
\newcommand{\FunctionTok}[1]{\textcolor[rgb]{0.02,0.16,0.49}{#1}}
\newcommand{\ImportTok}[1]{#1}
\newcommand{\InformationTok}[1]{\textcolor[rgb]{0.38,0.63,0.69}{\textbf{\textit{#1}}}}
\newcommand{\KeywordTok}[1]{\textcolor[rgb]{0.00,0.44,0.13}{\textbf{#1}}}
\newcommand{\NormalTok}[1]{#1}
\newcommand{\OperatorTok}[1]{\textcolor[rgb]{0.40,0.40,0.40}{#1}}
\newcommand{\OtherTok}[1]{\textcolor[rgb]{0.00,0.44,0.13}{#1}}
\newcommand{\PreprocessorTok}[1]{\textcolor[rgb]{0.74,0.48,0.00}{#1}}
\newcommand{\RegionMarkerTok}[1]{#1}
\newcommand{\SpecialCharTok}[1]{\textcolor[rgb]{0.25,0.44,0.63}{#1}}
\newcommand{\SpecialStringTok}[1]{\textcolor[rgb]{0.73,0.40,0.53}{#1}}
\newcommand{\StringTok}[1]{\textcolor[rgb]{0.25,0.44,0.63}{#1}}
\newcommand{\VariableTok}[1]{\textcolor[rgb]{0.10,0.09,0.49}{#1}}
\newcommand{\VerbatimStringTok}[1]{\textcolor[rgb]{0.25,0.44,0.63}{#1}}
\newcommand{\WarningTok}[1]{\textcolor[rgb]{0.38,0.63,0.69}{\textbf{\textit{#1}}}}
\usepackage{graphicx,grffile}
\makeatletter
\def\maxwidth{\ifdim\Gin@nat@width>\linewidth\linewidth\else\Gin@nat@width\fi}
\def\maxheight{\ifdim\Gin@nat@height>\textheight\textheight\else\Gin@nat@height\fi}
\makeatother
% Scale images if necessary, so that they will not overflow the page
% margins by default, and it is still possible to overwrite the defaults
% using explicit options in \includegraphics[width, height, ...]{}
\setkeys{Gin}{width=\maxwidth,height=\maxheight,keepaspectratio}
% Set default figure placement to htbp
\makeatletter
\def\fps@figure{htbp}
\makeatother
\setlength{\emergencystretch}{3em} % prevent overfull lines
\providecommand{\tightlist}{%
  \setlength{\itemsep}{0pt}\setlength{\parskip}{0pt}}
\setcounter{secnumdepth}{-\maxdimen} % remove section numbering

\date{}

\begin{document}

\hypertarget{header-n0}{%
\paragraph{题目}\label{header-n0}}

平面上有若干个点,现在要求用最少的底边在X轴上且面积小等A的矩形覆盖所有点,这些矩形可以重叠。
N\textless=100,A\textless=2000000

\hypertarget{header-n4}{%
\paragraph{思路}\label{header-n4}}

一开始想的是简单的区间dp。

\(f[l,r]\)表示覆盖完\([l,r]\)一段区间的所有点的最小矩形数,然后很快就发现了不对之处:

\begin{figure}
\centering
\includegraphics{https://i.loli.net/2019/07/21/5d3418fa286d466847.png}
\caption{}
\end{figure}

对于图中所示情况,单纯考虑区间之间的分割是行不通的,也就是说,对于相互重叠的矩形,高度那一维也很有必要记录。

重新定义状态:\(f[i][j][k]\)当\([i,j]\)区间,高度\(>k\)的点被覆盖的最优情况。

接下来考虑从上往下转移:

对于当前区间\([l,r,h]\),一种方式直接递归转移\([l+1,r,h]+1\)

另外一种方式,直接考虑从\(l\)开始,覆盖一定程度的点到\(j\),计算出最大能够达到的高度\(mxh\)

那么在\(mxh\)之上的点可以递归处理,式子为\(f[l][j][mxh]+f[j+1][r][h]+1\)。

\hypertarget{header-n14}{%
\paragraph{代码}\label{header-n14}}

\begin{Shaded}
\begin{Highlighting}[]
\PreprocessorTok{#include}\ImportTok{<bits/stdc++.h>}\PreprocessorTok{
}
\KeywordTok{using} \KeywordTok{namespace}\NormalTok{ std;
}
\DataTypeTok{void}\NormalTok{ tomin(}\DataTypeTok{int}\NormalTok{ &x,}\DataTypeTok{int}\NormalTok{ y)\{}\ControlFlowTok{if}\NormalTok{(x>y)x=y;\}
}
\DataTypeTok{void}\NormalTok{ tomax(}\DataTypeTok{int}\NormalTok{ &x,}\DataTypeTok{int}\NormalTok{ y)\{}\ControlFlowTok{if}\NormalTok{(x<y)x=y;\}
}
\KeywordTok{struct}\NormalTok{ node\{
}
    \DataTypeTok{int}\NormalTok{ x,y;
}
    \DataTypeTok{bool} \KeywordTok{operator}\NormalTok{ < (}\AttributeTok{const}\NormalTok{ node& res)}\AttributeTok{const}\NormalTok{\{
}
        \ControlFlowTok{if}\NormalTok{(x!=res.x)}\ControlFlowTok{return}\NormalTok{ x<res.x;
}
        \ControlFlowTok{return}\NormalTok{ y>res.y;  
}
\NormalTok{    \}
}
\NormalTok{\}A[}\DecValTok{105}\NormalTok{];
}
\DataTypeTok{int}\NormalTok{ n,S,f[}\DecValTok{105}\NormalTok{][}\DecValTok{105}\NormalTok{][}\DecValTok{105}\NormalTok{],mx[}\DecValTok{105}\NormalTok{][}\DecValTok{105}\NormalTok{],B[}\DecValTok{105}\NormalTok{],bc;
}
\DataTypeTok{int}\NormalTok{ mp[}\DecValTok{200005}\NormalTok{];
}
\DataTypeTok{int}\NormalTok{ dfs(}\DataTypeTok{int}\NormalTok{ l,}\DataTypeTok{int}\NormalTok{ r,}\DataTypeTok{int}\NormalTok{ h)\{
}
    \ControlFlowTok{if}\NormalTok{(mx[l][r]<=h)}\ControlFlowTok{return} \DecValTok{0}\NormalTok{;
}
    \ControlFlowTok{if}\NormalTok{(l==r)}\ControlFlowTok{return}\NormalTok{ f[l][r][h]=}\DecValTok{1}\NormalTok{;
}
    \ControlFlowTok{if}\NormalTok{(~f[l][r][h])}\ControlFlowTok{return}\NormalTok{ f[l][r][h];
}
    \DataTypeTok{int}\NormalTok{ L=l,R=r;
}
    \ControlFlowTok{while}\NormalTok{(L<=R&&A[L].y<=h)L++;
}
    \ControlFlowTok{while}\NormalTok{(L<=R&&A[R].y<=h)R--;
}
    \DataTypeTok{int}\NormalTok{ &res=f[l][r][h];res=dfs(L+}\DecValTok{1}\NormalTok{,R,h)+}\DecValTok{1}\NormalTok{;
}
    \ControlFlowTok{for}\NormalTok{(}\DataTypeTok{int}\NormalTok{ i=L+}\DecValTok{1}\NormalTok{;i<=R;i++)\{
}
        \DataTypeTok{int}\NormalTok{ d=A[i].x-A[L].x;
}
        \DataTypeTok{int}\NormalTok{ mxh=mp[S/d];
}
        \ControlFlowTok{if}\NormalTok{(mxh<=h)}\ControlFlowTok{break}\NormalTok{;
}
\NormalTok{        tomin(res,dfs(L,i,mxh)+dfs(i+}\DecValTok{1}\NormalTok{,R,h)+}\DecValTok{1}\NormalTok{); 
}
\NormalTok{    \}
}
    \ControlFlowTok{return}\NormalTok{ res;
}
\NormalTok{\}
}
\DataTypeTok{int}\NormalTok{ main()\{
}
\NormalTok{    memset(f,-}\DecValTok{1}\NormalTok{,}\KeywordTok{sizeof}\NormalTok{(f));
}
\NormalTok{    scanf(}\StringTok{"}\SpecialCharTok{%d%d}\StringTok{"}\NormalTok{,&n,&S);
}
    \ControlFlowTok{for}\NormalTok{(}\DataTypeTok{int}\NormalTok{ i=}\DecValTok{1}\NormalTok{;i<=n;i++)
}
\NormalTok{        scanf(}\StringTok{"}\SpecialCharTok{%d%d}\StringTok{"}\NormalTok{,&A[i].x,&A[i].y),B[++bc]=A[i].y;
}
\NormalTok{    sort(A+}\DecValTok{1}\NormalTok{,A+n+}\DecValTok{1}\NormalTok{);
}
    \DataTypeTok{int}\NormalTok{ len=}\DecValTok{1}\NormalTok{;
}
    \ControlFlowTok{for}\NormalTok{(}\DataTypeTok{int}\NormalTok{ i=}\DecValTok{2}\NormalTok{;i<=n;i++)
}
        \ControlFlowTok{if}\NormalTok{(A[i].x!=A[i-}\DecValTok{1}\NormalTok{].x)A[++len]=A[i];
}
\NormalTok{    n=len;sort(B+}\DecValTok{1}\NormalTok{,B+bc+}\DecValTok{1}\NormalTok{);bc=unique(B+}\DecValTok{1}\NormalTok{,B+bc+}\DecValTok{1}\NormalTok{)-B-}\DecValTok{1}\NormalTok{;
}
    \ControlFlowTok{for}\NormalTok{(}\DataTypeTok{int}\NormalTok{ i=}\DecValTok{1}\NormalTok{;i<=bc;i++)mp[B[i]]=i;
}
    \ControlFlowTok{for}\NormalTok{(}\DataTypeTok{int}\NormalTok{ i=}\DecValTok{1}\NormalTok{;i<=n;i++)
}
\NormalTok{        A[i].y=lower_bound(B+}\DecValTok{1}\NormalTok{,B+bc+}\DecValTok{1}\NormalTok{,A[i].y)-B;
}
    \ControlFlowTok{for}\NormalTok{(}\DataTypeTok{int}\NormalTok{ i=}\DecValTok{1}\NormalTok{;i<=}\DecValTok{200000}\NormalTok{;i++)
}
        \ControlFlowTok{if}\NormalTok{(!mp[i])mp[i]=mp[i-}\DecValTok{1}\NormalTok{];
}
    \ControlFlowTok{for}\NormalTok{(}\DataTypeTok{int}\NormalTok{ i=}\DecValTok{1}\NormalTok{;i<=n;i++)
}
        \ControlFlowTok{for}\NormalTok{(}\DataTypeTok{int}\NormalTok{ j=i,res=A[i].y;j<=n;j++)
}
\NormalTok{            tomax(res,A[j].y),mx[i][j]=res; 
}
\NormalTok{    printf(}\StringTok{"}\SpecialCharTok{%d\textbackslash{}n}\StringTok{"}\NormalTok{,dfs(}\DecValTok{1}\NormalTok{,n,}\DecValTok{0}\NormalTok{));
}
    \ControlFlowTok{return} \DecValTok{0}\NormalTok{;
}
\NormalTok{\}}
\end{Highlighting}
\end{Shaded}

\end{document}
